%--------------
%% Modified by Zhiqiang Lin
%% 01/18/2012
%%
%% Template file was from http://www.cs.umd.edu/~jkatz
%%
%% preamble.tex
%% this should be included with a command like
%% %--------------
%% Modified by Zhiqiang Lin
%% 01/18/2012
%%
%% Template file was from http://www.cs.umd.edu/~jkatz
%%
%% preamble.tex
%% this should be included with a command like
%% %--------------
%% Modified by Zhiqiang Lin
%% 01/18/2012
%%
%% Template file was from http://www.cs.umd.edu/~jkatz
%%
%% preamble.tex
%% this should be included with a command like
%% %--------------
%% Modified by Zhiqiang Lin
%% 01/18/2012
%%
%% Template file was from http://www.cs.umd.edu/~jkatz
%%
%% preamble.tex
%% this should be included with a command like
%% \input{preamble.tex}
%% \lecture{``lecture number''}{``date''}{``name of professor''}{``name
%%  of student''}

\hbadness=10000
\vbadness=10000

\newcommand{\handout}[5]{
   \renewcommand{\thepage}{#1-\arabic{page}}
   \noindent
   \begin{center}
   \framebox{
      \vbox{
    \hbox to 5.78in { {\bf CSE 5473 --- Network Security} 
     	 \hfill #2 }
       \vspace{4mm}
       \hbox to 5.78in { {\Large \hfill #5  \hfill} }
       \vspace{2mm}
       \hbox to 5.78in { {\it #3 \hfill #4} }
      }
   }
   \end{center}
   \vspace*{4mm}
}

\newcommand{\lecture}[4]{\handout{#1}{#2}{#3}{#4}{#1}}
%\newcommand{\lecture}[4]{\handout{#1}{#2}{Lecturer: #3}{Scribe(s): #4}{Lecture #1}}

\def\epsilon{\varepsilon}
\def\phi{\varphi}
\def\bool{\{0,1\}}
\def\poly{{\sf poly}}
\def\cross{\times}

\newcommand{\xor}{\oplus}
\newcommand{\Xor}{\bigoplus}
\newcommand{\ceil}[1]{\left\lceil {#1} \right\rceil}
\newcommand{\floor}[1]{\left\lfloor #1 \right\rfloor}
\newcommand{\ignore}[1]{}
\newcommand{\integers}[1]{{\mathbb Z}_{#1}}
\newcommand{\bydef}{\stackrel{\rm def}{=}}
\newcommand{\isequal}{\stackrel{\rm ?}{=}}
\newcommand{\compeq}{\stackrel{\rm c}{\equiv}} % computationally indistinguishable

\newcommand{\qed}{\hspace*{\fill}\rule{7pt}{7pt}}
\newenvironment{proof_sketch}{\noindent{\bf Sketch of Proof} (Informal)\hspace*{1em}}{\qed\medskip}
\newenvironment{proof}{\noindent{\bf Proof}\hspace*{1em}}{\qed\medskip}
\newenvironment{proofof}[1]{\noindent{\bf Proof} of #1:\hspace*{1em}}{\qed\medskip}
%\newenvironment{claim}{\noindent{\bf Claim}\hspace*{1em}\begin{em}}{\end{em}\medskip}
\newcounter{defcounter}
\setcounter{defcounter}{1}
\newenvironment{definition}{\medskip\noindent{\bf Definition \thedefcounter}}{\hspace*{\fill}$\diamondsuit$\stepcounter{defcounter}\medskip}
\newtheorem{theorem}{Theorem}
\newtheorem{corollary}[theorem]{Corollary}
\newtheorem{lemma}[theorem]{Lemma}
\newtheorem{claim}[theorem]{Claim}
\newtheorem{fact}[theorem]{Fact}
\newtheorem{conjecture}[theorem]{Conjecture}
\newenvironment{assumption}{\noindent{\bf Assumption}\hspace*{1em}\begin{em}}{\end{em}\medskip}
\newenvironment{remark}{\noindent{\bf Remark}\hspace*{1em}}{\bigskip}

\ignore{
\newcommand{\FOR}{{\bf for}}
\newcommand{\TO}{{\bf to}}
\newcommand{\DO}{{\bf do}}
\newcommand{\WHILE}{{\bf while}}
\newcommand{\AND}{{\bf and}}
\newcommand{\IF}{{\bf if}}
\newcommand{\THEN}{{\bf then}}
\newcommand{\ELSE}{{\bf else}}

%%% You probably will not need to use the commands listed below

\makeatletter
\def\fnum@figure{{\bf Figure \thefigure}}
\def\fnum@table{{\bf Table \thetable}}
\long\def\@mycaption#1[#2]#3{\addcontentsline{\csname
  ext@#1\endcsname}{#1}{\protect\numberline{\csname 
  the#1\endcsname}{\ignorespaces #2}}\par
  \begingroup
    \@parboxrestore
    \small
    \@makecaption{\csname fnum@#1\endcsname}{\ignorespaces #3}\par
  \endgroup}
\def\mycaption{\refstepcounter\@captype \@dblarg{\@mycaption\@captype}}
\makeatother

\newcommand{\figcaption}[1]{\mycaption[]{#1}}
\newcommand{\tabcaption}[1]{\mycaption[]{#1}}
\newcommand{\head}[1]{\chapter[Lecture \##1]{}}
\newcommand{\mathify}[1]{\ifmmode{#1}\else\mbox{$#1$}\fi}
\def\half{\frac{1}{2}}

\newcommand{\fig}[4]{
        \begin{figure}
        \setlength{\epsfysize}{#2}
        \vspace{3mm}
        \centerline{\epsfbox{#4}}
        \caption{#3} \label{#1}
        \end{figure}
        }
}



%% \lecture{``lecture number''}{``date''}{``name of professor''}{``name
%%  of student''}

\hbadness=10000
\vbadness=10000

\newcommand{\handout}[5]{
   \renewcommand{\thepage}{#1-\arabic{page}}
   \noindent
   \begin{center}
   \framebox{
      \vbox{
    \hbox to 5.78in { {\bf CSE 5473 --- Network Security} 
     	 \hfill #2 }
       \vspace{4mm}
       \hbox to 5.78in { {\Large \hfill #5  \hfill} }
       \vspace{2mm}
       \hbox to 5.78in { {\it #3 \hfill #4} }
      }
   }
   \end{center}
   \vspace*{4mm}
}

\newcommand{\lecture}[4]{\handout{#1}{#2}{#3}{#4}{#1}}
%\newcommand{\lecture}[4]{\handout{#1}{#2}{Lecturer: #3}{Scribe(s): #4}{Lecture #1}}

\def\epsilon{\varepsilon}
\def\phi{\varphi}
\def\bool{\{0,1\}}
\def\poly{{\sf poly}}
\def\cross{\times}

\newcommand{\xor}{\oplus}
\newcommand{\Xor}{\bigoplus}
\newcommand{\ceil}[1]{\left\lceil {#1} \right\rceil}
\newcommand{\floor}[1]{\left\lfloor #1 \right\rfloor}
\newcommand{\ignore}[1]{}
\newcommand{\integers}[1]{{\mathbb Z}_{#1}}
\newcommand{\bydef}{\stackrel{\rm def}{=}}
\newcommand{\isequal}{\stackrel{\rm ?}{=}}
\newcommand{\compeq}{\stackrel{\rm c}{\equiv}} % computationally indistinguishable

\newcommand{\qed}{\hspace*{\fill}\rule{7pt}{7pt}}
\newenvironment{proof_sketch}{\noindent{\bf Sketch of Proof} (Informal)\hspace*{1em}}{\qed\medskip}
\newenvironment{proof}{\noindent{\bf Proof}\hspace*{1em}}{\qed\medskip}
\newenvironment{proofof}[1]{\noindent{\bf Proof} of #1:\hspace*{1em}}{\qed\medskip}
%\newenvironment{claim}{\noindent{\bf Claim}\hspace*{1em}\begin{em}}{\end{em}\medskip}
\newcounter{defcounter}
\setcounter{defcounter}{1}
\newenvironment{definition}{\medskip\noindent{\bf Definition \thedefcounter}}{\hspace*{\fill}$\diamondsuit$\stepcounter{defcounter}\medskip}
\newtheorem{theorem}{Theorem}
\newtheorem{corollary}[theorem]{Corollary}
\newtheorem{lemma}[theorem]{Lemma}
\newtheorem{claim}[theorem]{Claim}
\newtheorem{fact}[theorem]{Fact}
\newtheorem{conjecture}[theorem]{Conjecture}
\newenvironment{assumption}{\noindent{\bf Assumption}\hspace*{1em}\begin{em}}{\end{em}\medskip}
\newenvironment{remark}{\noindent{\bf Remark}\hspace*{1em}}{\bigskip}

\ignore{
\newcommand{\FOR}{{\bf for}}
\newcommand{\TO}{{\bf to}}
\newcommand{\DO}{{\bf do}}
\newcommand{\WHILE}{{\bf while}}
\newcommand{\AND}{{\bf and}}
\newcommand{\IF}{{\bf if}}
\newcommand{\THEN}{{\bf then}}
\newcommand{\ELSE}{{\bf else}}

%%% You probably will not need to use the commands listed below

\makeatletter
\def\fnum@figure{{\bf Figure \thefigure}}
\def\fnum@table{{\bf Table \thetable}}
\long\def\@mycaption#1[#2]#3{\addcontentsline{\csname
  ext@#1\endcsname}{#1}{\protect\numberline{\csname 
  the#1\endcsname}{\ignorespaces #2}}\par
  \begingroup
    \@parboxrestore
    \small
    \@makecaption{\csname fnum@#1\endcsname}{\ignorespaces #3}\par
  \endgroup}
\def\mycaption{\refstepcounter\@captype \@dblarg{\@mycaption\@captype}}
\makeatother

\newcommand{\figcaption}[1]{\mycaption[]{#1}}
\newcommand{\tabcaption}[1]{\mycaption[]{#1}}
\newcommand{\head}[1]{\chapter[Lecture \##1]{}}
\newcommand{\mathify}[1]{\ifmmode{#1}\else\mbox{$#1$}\fi}
\def\half{\frac{1}{2}}

\newcommand{\fig}[4]{
        \begin{figure}
        \setlength{\epsfysize}{#2}
        \vspace{3mm}
        \centerline{\epsfbox{#4}}
        \caption{#3} \label{#1}
        \end{figure}
        }
}



%% \lecture{``lecture number''}{``date''}{``name of professor''}{``name
%%  of student''}

\hbadness=10000
\vbadness=10000

\newcommand{\handout}[5]{
   \renewcommand{\thepage}{#1-\arabic{page}}
   \noindent
   \begin{center}
   \framebox{
      \vbox{
    \hbox to 5.78in { {\bf CSE 5473 --- Network Security} 
     	 \hfill #2 }
       \vspace{4mm}
       \hbox to 5.78in { {\Large \hfill #5  \hfill} }
       \vspace{2mm}
       \hbox to 5.78in { {\it #3 \hfill #4} }
      }
   }
   \end{center}
   \vspace*{4mm}
}

\newcommand{\lecture}[4]{\handout{#1}{#2}{#3}{#4}{#1}}
%\newcommand{\lecture}[4]{\handout{#1}{#2}{Lecturer: #3}{Scribe(s): #4}{Lecture #1}}

\def\epsilon{\varepsilon}
\def\phi{\varphi}
\def\bool{\{0,1\}}
\def\poly{{\sf poly}}
\def\cross{\times}

\newcommand{\xor}{\oplus}
\newcommand{\Xor}{\bigoplus}
\newcommand{\ceil}[1]{\left\lceil {#1} \right\rceil}
\newcommand{\floor}[1]{\left\lfloor #1 \right\rfloor}
\newcommand{\ignore}[1]{}
\newcommand{\integers}[1]{{\mathbb Z}_{#1}}
\newcommand{\bydef}{\stackrel{\rm def}{=}}
\newcommand{\isequal}{\stackrel{\rm ?}{=}}
\newcommand{\compeq}{\stackrel{\rm c}{\equiv}} % computationally indistinguishable

\newcommand{\qed}{\hspace*{\fill}\rule{7pt}{7pt}}
\newenvironment{proof_sketch}{\noindent{\bf Sketch of Proof} (Informal)\hspace*{1em}}{\qed\medskip}
\newenvironment{proof}{\noindent{\bf Proof}\hspace*{1em}}{\qed\medskip}
\newenvironment{proofof}[1]{\noindent{\bf Proof} of #1:\hspace*{1em}}{\qed\medskip}
%\newenvironment{claim}{\noindent{\bf Claim}\hspace*{1em}\begin{em}}{\end{em}\medskip}
\newcounter{defcounter}
\setcounter{defcounter}{1}
\newenvironment{definition}{\medskip\noindent{\bf Definition \thedefcounter}}{\hspace*{\fill}$\diamondsuit$\stepcounter{defcounter}\medskip}
\newtheorem{theorem}{Theorem}
\newtheorem{corollary}[theorem]{Corollary}
\newtheorem{lemma}[theorem]{Lemma}
\newtheorem{claim}[theorem]{Claim}
\newtheorem{fact}[theorem]{Fact}
\newtheorem{conjecture}[theorem]{Conjecture}
\newenvironment{assumption}{\noindent{\bf Assumption}\hspace*{1em}\begin{em}}{\end{em}\medskip}
\newenvironment{remark}{\noindent{\bf Remark}\hspace*{1em}}{\bigskip}

\ignore{
\newcommand{\FOR}{{\bf for}}
\newcommand{\TO}{{\bf to}}
\newcommand{\DO}{{\bf do}}
\newcommand{\WHILE}{{\bf while}}
\newcommand{\AND}{{\bf and}}
\newcommand{\IF}{{\bf if}}
\newcommand{\THEN}{{\bf then}}
\newcommand{\ELSE}{{\bf else}}

%%% You probably will not need to use the commands listed below

\makeatletter
\def\fnum@figure{{\bf Figure \thefigure}}
\def\fnum@table{{\bf Table \thetable}}
\long\def\@mycaption#1[#2]#3{\addcontentsline{\csname
  ext@#1\endcsname}{#1}{\protect\numberline{\csname 
  the#1\endcsname}{\ignorespaces #2}}\par
  \begingroup
    \@parboxrestore
    \small
    \@makecaption{\csname fnum@#1\endcsname}{\ignorespaces #3}\par
  \endgroup}
\def\mycaption{\refstepcounter\@captype \@dblarg{\@mycaption\@captype}}
\makeatother

\newcommand{\figcaption}[1]{\mycaption[]{#1}}
\newcommand{\tabcaption}[1]{\mycaption[]{#1}}
\newcommand{\head}[1]{\chapter[Lecture \##1]{}}
\newcommand{\mathify}[1]{\ifmmode{#1}\else\mbox{$#1$}\fi}
\def\half{\frac{1}{2}}

\newcommand{\fig}[4]{
        \begin{figure}
        \setlength{\epsfysize}{#2}
        \vspace{3mm}
        \centerline{\epsfbox{#4}}
        \caption{#3} \label{#1}
        \end{figure}
        }
}



%% \lecture{``lecture number''}{``date''}{``name of professor''}{``name
%%  of student''}

\hbadness=10000
\vbadness=10000

\newcommand{\handout}[5]{
   \renewcommand{\thepage}{#1-\arabic{page}}
   \noindent
   \begin{center}
   \framebox{
      \vbox{
    \hbox to 5.78in { {\bf CSE 5473 --- Network Security}
     	 \hfill #2 }
       \vspace{4mm}
       \hbox to 5.78in { {\Large \hfill #5  \hfill} }
       \vspace{2mm}
       \hbox to 5.78in { {\it #3 \hfill #4} }
      }
   }
   \end{center}
   \vspace*{4mm}
}

\newcommand{\lecture}[4]{\handout{#1}{#2}{#3}{#4}{#1}}
%\newcommand{\lecture}[4]{\handout{#1}{#2}{Lecturer: #3}{Scribe(s): #4}{Lecture #1}}

\def\epsilon{\varepsilon}
\def\phi{\varphi}
\def\bool{\{0,1\}}
\def\poly{{\sf poly}}
\def\cross{\times}

\newcommand{\xor}{\oplus}
\newcommand{\Xor}{\bigoplus}
\newcommand{\ceil}[1]{\left\lceil {#1} \right\rceil}
\newcommand{\floor}[1]{\left\lfloor #1 \right\rfloor}
\newcommand{\ignore}[1]{}
\newcommand{\integers}[1]{{\mathbb Z}_{#1}}
\newcommand{\bydef}{\stackrel{\rm def}{=}}
\newcommand{\isequal}{\stackrel{\rm ?}{=}}
\newcommand{\compeq}{\stackrel{\rm c}{\equiv}} % computationally indistinguishable

\newcommand{\qed}{\hspace*{\fill}\rule{7pt}{7pt}}
\newenvironment{proof_sketch}{\noindent{\bf Sketch of Proof} (Informal)\hspace*{1em}}{\qed\medskip}
\newenvironment{proof}{\noindent{\bf Proof}\hspace*{1em}}{\qed\medskip}
\newenvironment{proofof}[1]{\noindent{\bf Proof} of #1:\hspace*{1em}}{\qed\medskip}
%\newenvironment{claim}{\noindent{\bf Claim}\hspace*{1em}\begin{em}}{\end{em}\medskip}
\newcounter{defcounter}
\setcounter{defcounter}{1}
\newenvironment{definition}{\medskip\noindent{\bf Definition \thedefcounter}}{\hspace*{\fill}$\diamondsuit$\stepcounter{defcounter}\medskip}
\newtheorem{theorem}{Theorem}
\newtheorem{corollary}[theorem]{Corollary}
\newtheorem{lemma}[theorem]{Lemma}
\newtheorem{claim}[theorem]{Claim}
\newtheorem{fact}[theorem]{Fact}
\newtheorem{conjecture}[theorem]{Conjecture}
\newenvironment{assumption}{\noindent{\bf Assumption}\hspace*{1em}\begin{em}}{\end{em}\medskip}
\newenvironment{remark}{\noindent{\bf Remark}\hspace*{1em}}{\bigskip}

\ignore{
\newcommand{\FOR}{{\bf for}}
\newcommand{\TO}{{\bf to}}
\newcommand{\DO}{{\bf do}}
\newcommand{\WHILE}{{\bf while}}
\newcommand{\AND}{{\bf and}}
\newcommand{\IF}{{\bf if}}
\newcommand{\THEN}{{\bf then}}
\newcommand{\ELSE}{{\bf else}}

%%% You probably will not need to use the commands listed below

\makeatletter
\def\fnum@figure{{\bf Figure \thefigure}}
\def\fnum@table{{\bf Table \thetable}}
\long\def\@mycaption#1[#2]#3{\addcontentsline{\csname
  ext@#1\endcsname}{#1}{\protect\numberline{\csname
  the#1\endcsname}{\ignorespaces #2}}\par
  \begingroup
    \@parboxrestore
    \small
    \@makecaption{\csname fnum@#1\endcsname}{\ignorespaces #3}\par
  \endgroup}
\def\mycaption{\refstepcounter\@captype \@dblarg{\@mycaption\@captype}}
\makeatother

\newcommand{\figcaption}[1]{\mycaption[]{#1}}
\newcommand{\tabcaption}[1]{\mycaption[]{#1}}
\newcommand{\head}[1]{\chapter[Lecture \##1]{}}
\newcommand{\mathify}[1]{\ifmmode{#1}\else\mbox{$#1$}\fi}
\def\half{\frac{1}{2}}

\newcommand{\fig}[4]{
        \begin{figure}
        \setlength{\epsfysize}{#2}
        \vspace{3mm}
        \centerline{\epsfbox{#4}}
        \caption{#3} \label{#1}
        \end{figure}
        }
}
