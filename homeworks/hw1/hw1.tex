\documentclass[11pt]{article}
%\usepackage{psfig}
\usepackage{latexsym}
\usepackage{amsfonts}
\usepackage{hyperref}
\usepackage{url}
\usepackage{listings}
% to use hyperlinks in References section
\hypersetup{urlbordercolor={1 1 1}}
\usepackage{graphicx}
\usepackage{listings}

%define colors
\usepackage{xcolor}
\definecolor{dkgreen}{rgb}{0,0.6,0}
\definecolor{gray}{rgb}{0.5,0.5,0.5}
\definecolor{mauve}{rgb}{0.58,0,0.82}
\usepackage{graphicx}
\usepackage{hyperref}
\hypersetup{urlbordercolor={1 1 1}}
\DeclareGraphicsExtensions{.pdf,.jpg,.png,.eps}
\setlength{\textheight}{8.5in}
\setlength{\textwidth}{6.0in}
\setlength{\headheight}{0in}
\addtolength{\topmargin}{-.5in}
\addtolength{\oddsidemargin}{-.5in}


\newenvironment{cvl}{\begin{list}{$\bullet$}{
\setlength{\leftmargin}{0.3in} \setlength{\labelsep}{0.07in}
\setlength{\labelwidth}{0.17in} \setlength{\rightmargin}{0.0in}
\setlength{\topsep}{0.000in} \setlength{\partopsep}{0.000in}
\setlength{\parskip}{0.000in} \setlength{\parsep}{0.005in}
\setlength{\itemsep}{0.005in}}}{\end{list}}

\DeclareGraphicsExtensions{.pdf,.jpg,.png,.eps}
\usepackage[T1]{fontenc}   %special characters copy-able

% \usepackage{times}
\usepackage{enumerate}

\usepackage{courier}
\usepackage{amsmath}
\usepackage{amssymb}

\input{preamble.tex}

\begin{document}


\lecture{Assignment \#1}{Autumn 2020}{Symmetric Cryptography}{Due September 23th, 11:59PM}

\lstset{ %
   language=python,                % the language of the code
   basicstyle=\footnotesize\ttfamily,           % the size of the fonts that are used for the code
   literate={-}{-}1,
   columns=fullflexible,   %copy-paste-able
   %numbers=left,                   % where to put the line-numbers
   numberstyle=\tiny\color{gray},  % the style that is used for the line-numbers
   stepnumber=2,                   % the step between two line-numbers. If it's 1, each line
                                   % will be numbered
   numbersep=5pt,                  % how far the line-numbers are from the code
   backgroundcolor=\color{white},      % choose the background color. You must add \usepackage{color}
   showspaces=false,               % show spaces adding particular underscores
   showstringspaces=false,         % underline spaces within strings
   showtabs=false,                 % show tabs within strings adding particular underscores
   frame=single,                   % adds a frame around the code
   rulecolor=\color{black},        % if not set, the frame-color may be changed on line-breaks within not-black text (e.g. commens (green here))
   tabsize=2,                      % sets default tabsize to 2 spaces
   captionpos=b,                   % sets the caption-position to bottom
   breaklines=true,                % sets automatic line breaking
   breakatwhitespace=false,        % sets if automatic breaks should only happen at whitespace
   title=\lstname,                   % show the filename of files included with \lstinputlisting;
                                   % also try caption instead of title
   keywordstyle=\color{blue},          % keyword style
   commentstyle=\color{dkgreen},       % comment style
   stringstyle=\color{mauve},         % string literal style
   escapeinside={\%*}{*)},            % if you want to add LaTeX within your code
   morekeywords={*,...}               % if you want to add more keywords to the set
 }
%%%% body goes in here %%%%






\section{Secret Key Cryptography - DES [30 pts]}
\begin{enumerate}
\item Search the web for a DES implementation (e.g.,  \url{http://des.online-domain-tools.com/}) and use it to find the result of encrypting the plaintext {\tt A23B716BA14C32C9} with the key {\tt D43CB19AF490D7CE} in hexadecimal format \\

Initialization Vector (IV): {\tt 00 31 0d 31 f8 df 90 df}
  \begin{enumerate}
    \item des-ecb: {\tt 08 83 53 71 26 38 52 45 8e 1e f7 96 c2 d3 0c 12}
    \item des-cbc (with IV): {\tt 1b 9f 37 6d be 78 fe 0a 4c 6c 02 d8 39 9d c2 73}
    \item des-cfb (with IV): {\tt d9 eb 3b 13 9d 5a 5a 70 3f 0a 8a ea 26 de a6 9f}
    \item des-ofb (with IV): {\tt d9 3a e8 55 58 76 7d 37 be 4d 60 44 37 b2 3e e3}
    \item des-nofb (with IV): {\tt d9 20 00 6e 71 a1 5a 68 77 19 f0 d0 72 27 1d de}
  \end{enumerate}

\item The keys {\tt 0000000000000000} and {\tt FFFFFFFFFFFFFFFF} (in hexadecimal format) are considered ``weak'' keys for DES (\url{http://en.wikipedia.org/wiki/Weak_key}). Explain why these keys are weak. (Hint: Consider the per-round keys that they generate and the effect these per-round keys have on each round of the algorithm --  You might want to check your textbook or do a google search to understand what a  ``weak'' key means.)

When we are using DES algorithm to encrypt a message, DES is going to take the plaintext $m$ with sixteen round Feistel network to get the ciphertext $c$. To be more specific, after the Initial Permutation ($IP$), the 64-bit message will be equally splited to two parts, $L_{i-1}$ and $R_{i-1}$. Then, DES will following function to encrypt the message:

\begin{align*}
  L_{i}&=R_{i-1} \\
  R_{i}&=L{i-1}\xor f(R_{i-1}, K_{i})
\end{align*}

The Key $k$ is not being used directly, instead, there is a Key Schedule which generate 16 Subkeys. The 64-bit $k$ will go through Permutated Choice 1 ($PC1$), and output two 28-bit key $C_{i}$ and $D_{i}$ (every 8th bit in the key is not used). Then DES will use Permutated Choice 2 ($PC2$) to get a 48-bit key $K_{i}$ ($LS$ means Left Shift):

\begin{align*}
  C_{i}=LS_{i}(C_{i}-1) \\
  D_{i}=LS_{i}(D_{i}-1) \\
  K_{i}=PC2(C_{i}D_{i})
\end{align*}

If we are using keys like {\tt 0000000000000000} and {\tt FFFFFFFFFFFFFFFF} through $PC1$, the value of $C_{i}D_{i}$ will be all 0 or all 1 and casue $K_{i}=K_{15-i}$ (Subkeys are symmetric). Since DES is using reversing function to decrypt, which means we can use same $k$ to encrypt the ciphertext $c$ again to get the plaintext $m$.

\item If DES keys had been 128 bits as IBM originally proposed, how long would one million computers, each trying one trillion keys per second, take to examine all possible keys?

DES is using 64-bit key, but only 56-bit has been used to encryption because every 8th bit in the key is not used. Once we increased the key length to 128 bits, there is only 112-bit can be used to encryption. Let's be clear, 1 million = $10^{6}$ and 1 trillion = $10^{12}$.

\begin{align*}
  \frac{2^{128}\text{ keys}}{10^{12}\text{ keys/second}}/10^{6}\text{ computer}&=3.40282 \times 10^{20} \text{ second/computer} \\
  &=5.67137 \times 10^{18} \text{ minute/computer} \\
  &=9.45229 \times 10^{16} \text{ hour/computer} \\
  &=3.93845 \times 10^{15} \text{ day/computer} \\
  &=1.07903 \times 10^{13} \text{ year/computer} \\
\end{align*}

\end{enumerate}

\section{Encrypting Large Messages [40 pts]}
\begin{enumerate}
\item Suppose a sequence of plaintext blocks, $x_1$; $x_2$; ...; $x_n$ yields the ciphertext sequence $y_1$; $y_2$; ...; $y_n$. Suppose that one ciphertext block, say $y_i$, is transmitted incorrectly (i.e., some 1's are changed to 0's and vice versa). What is the number of plaintext blocks that will be decrypted incorrectly when the following modes are used: (1) ECB, (2) CBC, (3) OFB, and (4) CFB modes are used.

  \begin{enumerate}
    \item ECB: 1 block
    \item CBC: 2 blocks
    \item OFB: 1 block
    \item CFB: 3 blocks
  \end{enumerate}

\clearpage

\item What pseudo-random block stream is generated by 64-bit OFB with a weak DES key?

When we are using OFB, we will use the output from pervious block stream to encrypt the next block stream. The sequence would be $E_{i}(IV), E_{i}(E_{i}(IV)), E_{i}(E_{i}(E_{i}(IV))), \cdots$. As we discussed before in section 1, week keys are symmetric, and we can get the plaintext by using the same algorithm to encrypt the ciphertext again. Therefore, the pseudo-random block stream generated by 64-bit OFB with a weak DES key would be $E_{i}(IV), IV, E_{i}(IV), IV, \ldots$

\item The pseudo-random stream of blocks generated by 64-bit OFB must eventually repeat (since at most $2^{64}$ different blocks can be generated). Will $k\{IV\}$ necessarily be the first block to be repeated? Explain.

$k\{IV\}$ is necessarily be the first block to be repeated since OFB will using the sequence $E_{i}(IV), E_{i}(E_{i}(IV)), E_{i}(E_{i}(E_{i}(IV))), \cdots$ to encrypt blocks which means the first block stream is used to encrypt the next block stream (and the next block stream is used to decrypt the pervious block stream).

\item A common message integrity check is DES-CBC, which means encrypting the message in CBC mode and using the final ciphertext block as the checksum. This final block is known as the ``CBC residue''. This is what is used in the banking industry. Show how you can construct a message with a particular CBC residue, with the only constraint that somewhere in the message you have to be able to embed 64 bits of ``garbage''.

We can leave the last block empty, and working from the last block (CBC residue) to the first block to find the bit who must be exist in order to maintain the CBC residue correct.

\item Consider the following alternative method to CBC for encrypting large messages. In this CBC variant, the encryption is done as shown in the Figure below. Is this variant reversible (can you decrypt the ciphertext generated by this scheme)? How? What are the security implications of this scheme, if any, as opposed to the ``normal'' CBC? (Hint: Think about passive and active attacks that are possible to carry on this scheme.)
\end{enumerate}

\begin{figure}[h!]
\includegraphics[width=6in]{cbc_variant.png}
\centering
\end{figure}

\begin{enumerate}
  \item Use $c_{1}\xor IV$ to get $m_{1}$.
  \item Use $m_{1}\xor c_{2}$ to get $m_{2}$.
  \item Use $m_{2}\xor c_{3}$ to get $m_{3}$.
  \item $\cdots$
  \item Use $m_{i}\xor c_{i+1}$ to get $m_{i+1}$.
\end{enumerate}

\begin{itemize}
  \item It is very easy to find $m_{i}$ if we have $m_{i+1}$ and $c_{i}+1$ which means the entire message will be compromised when only one of the block compromised.
  \item For passive attackers, if we know the plaintext and ciphertext, we can break the encryption easily by solving simple math problem.
\end{itemize}

\section{Multiple Encryption DES [30 pts]}
In answering this question, assume that an average close to $2^{32}$ steps to break an encryption system (find the corresponding key(s)) is tractable, while an average close to $2^{64}$ steps or larger is not.

\begin{enumerate}
\item Let's assume that you do DES double encryption by encrypting with $k_1$ and doing DES in decrypt mode with $k_2$. Does the same meet-in-the-middle attack described in the textbook work as with double encryption with $k_1$ and $k_2$? If not, how could it be made to work? (Hint: it is possible but requires some tweaks. Please describe your tweaks)

The meet-in-the-middle attack works.

\item Devise a meet-in-the-middle attack to break EDE. That is, find the keys $k_1$ and $k_2$ assuming knowledge of a set of plaintext/ciphertext pairs \textless$m_i, c_i$\textgreater obtained by applying EDE on each $p_i$.  (Hint: you meet either after ED, or before DE)

  \begin{enumerate}
    \item Generate all possible keys $k_{1\ldots 56}$, there are $2^{56}$ values.
    \item Use the $k_{i}$ to encrypt the first block of the plaintext $m_{1}$.
    \item Use the $k_{j}$ to decrypt the first block of the ciphertext $c_{1}$.
    \item If there is a key $k$ when $Enc(k, m_{1})=Dec(k, c_{1})$, store it for further verification.
    \item Test all values from pervious step, find the only value who meets $Enc(m_{i})=Dec(c_{i})$.
  \end{enumerate}

\item Assume that EDE was deployed using three keys $k_1$ (for step 1 encryption), $k_2$ (for step 2 decryption), and $k_3$ (for step 3 encryption). Is this new version of EDE more secure than the traditional EDE? Explain.

  \begin{enumerate}
    \item Assume $k_{1}\neq k_{2}\neq k_{3}$.
    \item Generate all possible keys $k_{1\ldots 56}$, there are $2^{56}$ values.
    \item Use the $k_{x}$ to decrypt the first block of the plaintext $m_{1}$.
    \item Use the $k_{y}$ to decrypt the first block of the ciphertext $c_{1}$.
    \item If $Dec(k_{x}, m_{1})=Dec(k_{y}, c_{1})$, store $k_{x}, k_{y}$.
    \item Use the $k_{z}$ to encrypt $Dec(k_{x}, m_{1})\mid Dec(k_{y}, c_{1})$, store all of them.
    \item Test all values from pervious step, find the plaintext.
  \end{enumerate}

  EDE has same security as the tridational EDE, since they have same complexity $O\left(n^{112}\right)$. \\

  For triditional EDE, we need to use $O\left(n^{56}\right)$ to solve the outer layer of encryption, and use $O\left(n^{112}\right)$ to get the plaintext. \\

  The new version of EDE has same security with triditional, since we need to take $O\left(2 \times n^{56}\right)$ to solve the outer layers, and use $O\left(n^{112}\right)$ to get the plaintext from middle layer with the outer layer.

\item Consider the following approach to increase the security of DES. Encryption is done by encrypting the plaintext twice using the same shared secret. Decryption is done by decrypting the ciphertext twice using the same shared secret. Is this approach more secure than the traditional DES?
\end{enumerate}

Encrypt twice with same shared secret will not secure than the traditional DES because we have demonstrated the meet-in-the-middle attack in question 3.2. We can use the same key $k_{i}$ to encrypt the plaintext and decrypt the ciphertext then get the value of $k_{i}$ which takes same complexity.

\section{Breaking Monoaphabetic Cipher Using Frequence Analysis [30 points Bonus. There will be partial credit if you only partially solve it.]}

A monoalphabetic cipher is one where each symbol in the input (called the ``plaintext'') is mapped to a fixed symbol in the output (called the ``ciphertext''). Let's go back to the age where we use statistic analysis (or frequence analysis) to break ciphers (esentially the keys).  It will be a fun assignment, and you can use modern techniques to break old ciphers.

While you can entirely finish this assignment by using pencil and paper, it is more efficient to use (existing) programs. Please \textbf{Find}\footnote{For instance, \url{http://crypto.interactive-maths.com/frequency-analysis-breaking-the-code.html}, which I recommend especially if you don't wanna write code.} or \textbf{write} a program\footnote{Please note that you don't have to write your program in our VM. You can use whatever computing environment convenient to you, e.g., Windows, Linux, Mac, etc., using Java, C, Python, etc.} that reads an input ciphertext, and outputs the (sorted) letter frequencies, diagram (consecutive pairs of letters) frequencies, and trigram (consecutive triplets of letters) frequencies.

Using this frequence analysis program, decode the following ciphertext. The original plaintext that produced this ciphertext contains only upper case letters;  all punctuation and spaces were removed before encoding.  It was encrypted with a mono-alphabetic cipher.  Explain how you decoded the ciphertext in your answer.

\begin{quote}
\raggedright{
{\tt
WMTLXJAVQNXBTECJIZMWDMEIGXXVTSXRAWHXHTNHVMTVW
KVMXBTECQGXUCXSWXNDXWYNQZQGNQZTIQGPAHRWNRQGVM
XGAVASXJAVJIVMXCTKVTNHPAHRWNRJIVMXCTKVWZWKMVQ
YNQZZMTVVMXSXMTKJXXNWNVMXDQNHADVQGVMXJSWVWKME
WNWKVSIGQSVMXBTKVVZQIXTSKVQPAKVWGIVMQKXMQCXKZ
WVMZMWDMRXNVBXEXNMTLXJXXNCBXTKXHVQKQBTDXVMXEK
XBLXKTNHVMXMQAKXWKWVVMTVWNKWHWQAKKEWBXZWVMZMW
DMQASCXVWVWQNMTKJXXNBTVXBISXDXWLXH
}
}
\end{quote}

\begin{quote}
\raggedright{
{\tt
IHAVEBUTONELAMPBYWHICHMYFEETAREGUIDEDANDTHATI
STHELAMPOFEXPERIENCEIKNOWOFNOWAYOFJUDGINGOFTH
EFUTUREBUTBYTHEPASTANDJUDGINGBYTHEPASTIWISHTO
KNOWWHATTHEREHASBEENINTHECONDUCTOFTHEBRITISHM
INISTRYFORTHELASTTWOYEARSTOJUSTIFYTHOSEHOPESW
ITHWHICHGENTLEMENHAVEBEENPLEASEDTOSOLACETHEMS
ELVESANDTHEHOUSEISITTHATINSIDIOUSSMILEWITHWHI
CHOURPETITIONHASBEENLATELYRECEIVED
}
}
\end{quote}

\begin{lstlisting}[language=Python]
str = """
WMTLXJAVQNXBTECJIZMWDMEIGXXVTSXRAWHXHTNHVMTVW
KVMXBTECQGXUCXSWXNDXWYNQZQGNQZTIQGPAHRWNRQGVM
XGAVASXJAVJIVMXCTKVTNHPAHRWNRJIVMXCTKVWZWKMVQ
YNQZZMTVVMXSXMTKJXXNWNVMXDQNHADVQGVMXJSWVWKME
WNWKVSIGQSVMXBTKVVZQIXTSKVQPAKVWGIVMQKXMQCXKZ
WVMZMWDMRXNVBXEXNMTLXJXXNCBXTKXHVQKQBTDXVMXEK
XBLXKTNHVMXMQAKXWKWVVMTVWNKWHWQAKKEWBXZWVMZMW
DMQASCXVWVWQNMTKJXXNBTVXBISXDXWLXH
""".replace("\n", "")

# find most common character
def check_freq(x):
    freq = {}
    for c in x:
        freq[c] = str.count(c)
    return freq


check_freq(str)

# abcdefghijklmnopqrstuvwxyz
# TJDHXGRMWPYBENQCFSKVALZUIO

# found 47 "X", replace as "e"
text = str.replace("X", "e")

# try to find most common double "e" to identify the word
# found mutiple "JeeN", suspect can be replaced by "been"
text = text.replace("J", "b").replace("N", "n")

# found mutiple "MTK" before "been"
text = text.replace("M", "h").replace("T", "a").replace("K", "s")

# found mutiple "haLe", only "L" did not be replaced, replace it as "v"
text = text.replace("L", "v")

# found "W", "have", replace "W" as "I"
text = text.replace("W", "i")

# found mutiple "Vhe"
text = text.replace("V", "t")

# found "bAt", replace "A" as "u"
text = text.replace("A", "u")

# found "Qne", replace "Q" as "o"
text = text.replace("Q", "o")

# found "DonHuDt", replace "D" as "c", "H" as "d"
text = text.replace("D", "c").replace("H", "d")

# found mutiple "oG", replace "G" as "f"
text = text.replace("G", "f")

# found "futuSe", replace "S" as "r"
text = text.replace("S", "r")

# found "bI", replace "I" as "y"
text = text.replace("I", "y")

# found "Zhich", replace "Z" as "w"
text = text.replace("Z", "w")

# found "Ey feet", replace "E" as "m"
text = text.replace("E", "m")

# found "wishtoYnow", replace "Y" as "k"
text = text.replace("Y", "k")

# found "BamC", replace "B" as "l", "C" as "p"
text = text.replace("B", "l").replace("C", "p")

# found "Ruided", replace "R" as "g"
text = text.replace("R", "g")

# found "Pudging", replace "P" as "j"
text = text.replace("P", "j")

# found "eUperience", replace "U" as "x"
text = text.replace("U", "x")

print(text)

# split by knowing words
rep = [
      "have",
      "what",
      "feet",
      "been",
      "house",
      "which",
      "with",
      "wish",
      "and",
      "know",
      "conduct",
      "but",
      "one",
      "for",
      "last",
      "to",
      "of",
      "years",
      "there",
      "them",
      "way",
      "those",
      "are",
      "that",
      "received",
      "insidious",
      "british",
      "by",
      ]

for item in rep:
    text = text.replace(item, " {} ".format(item))

" ".join(text.split())
\end{lstlisting}

\vspace{0.5in}
Reference \url{http://en.wikipedia.org/wiki/Frequency_analysis}
\vspace{0.5in}


\section{Submiting your report}
Please write a report describing how you solve each of the problem above. You can use WORD, or Latex to write the report. The Latex source of this assignement is available at \url{https://www.overleaf.com/read/xfwppdqjbzbt}. Regarding how to use Latex, please see this tutorial if you have 0-experience with it \url{https://www.andy-roberts.net/writing/latex/absolute_beginners}. Please note that latex can be installed in all modern OSes such as Windows, Linux, and Mac.

\end{document}
